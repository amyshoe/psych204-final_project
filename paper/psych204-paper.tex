% 
% Annual Cognitive Science Conference
% Sample LaTeX Paper -- Proceedings Format
% 

% Original : Ashwin Ram (ashwin@cc.gatech.edu)       04/01/1994
% Modified : Johanna Moore (jmoore@cs.pitt.edu)      03/17/1995
% Modified : David Noelle (noelle@ucsd.edu)          03/15/1996
% Modified : Pat Langley (langley@cs.stanford.edu)   01/26/1997
% Latex2e corrections by Ramin Charles Nakisa        01/28/1997 
% Modified : Tina Eliassi-Rad (eliassi@cs.wisc.edu)  01/31/1998
% Modified : Trisha Yannuzzi (trisha@ircs.upenn.edu) 12/28/1999 (in process)
% Modified : Mary Ellen Foster (M.E.Foster@ed.ac.uk) 12/11/2000
% Modified : Ken Forbus                              01/23/2004
% Modified : Eli M. Silk (esilk@pitt.edu)            05/24/2005
% Modified : Niels Taatgen (taatgen@cmu.edu)         10/24/2006
% Modified : David Noelle (dnoelle@ucmerced.edu)     11/19/2014

%% Change "letterpaper" in the following line to "a4paper" if you must.

\documentclass[10pt,letterpaper]{article}

\usepackage{cogsci}
\usepackage{pslatex}
\usepackage{apacite}


\title{Witty Lead-in: Learning Algebra Problem-Solving Strategies}
 
\author{{\large \bf Luis Apolaya (lapolaya@stanford.edu)} \\
  Department of Symbolic Systems, Stanford University
  \AND {\large \bf Amy Shoemaker (amyshoe@stanford.edu)} \\
  Institute for Computational and Mathematical Engineering, Stanford University}


\begin{document}

\maketitle


\begin{abstract}
The abstract should be one paragraph, indented 1/8~inch on both sides,
in 9~point font with single spacing. The heading ``{\bf Abstract}''
should be 10~point, bold, centered, with one line of space below
it. This one-paragraph abstract section is required only for standard
six page proceedings papers. Following the abstract should be a blank
line, followed by the header ``{\bf Keywords:}'' and a list of
descriptive keywords separated by semicolons, all in 9~point font, as
shown below.

\textbf{Keywords:} 
Mathematics Education; Mathematical Cognition; Probabilistic Models
\end{abstract}


\section{Introduction}

Content-wise, here we have two identical problems:

\begin{enumerate}
\item[(A)] Solve for $x$:\\ $x * 6 + 66 = 81.9$

\item[(B)] When Ted got home from his waiter job, he multiplied his hourly wage by the 6 hours he worked that day. Then he added the \$66.00 he made in tips and found he had earned \$81.90. How much does Ted make per hour?
\end{enumerate}

Though questions (A) and (B) give and ask for the same information, much can be said about how the presentations of these two problems affect how students and teachers diagnose their respective difficulty level, and therefore how likely they are to reach the correct answer. When asked which format best facilitates acquisition of algebraic concepts, for example, teachers agree that the first of the two problems above will do so better than the story problem because symbolic problems are inherently easier, and therefore ought to be taught first in an introductory algebra class (\textbf{citation}). The present paper takes a study conducted by (\textbf{citation}) that shines a light on situations where story problems are actually \textit{easier} than symbolic problems and, through the use of a probabilistic programming language, seeks to understand the cognitive mechanisms that bring forth the patterns Nathan and Koedinger noticed. As such, the authors hope to bring further light into the psychology of math students and assess how pedagogical methods reflecting the insights discussed in this paper can better attend to their development as problem solvers.


\subsection{Story Problems and the Verbal Hypothesis}

There are some reasons why mathematics teachers widely believe that symbolic representations of algebra problems are inherently easier than story problems. For one, symbolic problems are comprised of more easily manipulable components than their verbal counterparts, as all the relevant parts of the problem are present in the symbols and students do not have to identify the key parts of the problem from within a story context (\textbf{citation?}). On the other hand, verbal problems can ground the scenario in a more relatable, real-life manner, which may heighten students’ intuition or provide insight into informal problem-solving strategies they can use. Though surveys cited in \citeA{Nathan2012} show that teachers think symbolic problems are likely to be solved correctly, \citeA{KoedNath2004} show a disparity in the teachers' beliefs and students' performance, as  subjects in their study show a higher rate of success for verbal problems.

In explaining this phenomenon, \citeA{KoedNath2004} propose what they call the \textit{verbal hypothesis}, which states that the earlier acquisition of language gives students an advantage on comprehending and solving story problems over symbolic problems. That is, since the symbols that comprise an algebraic equation are to be understood and manipulated to reach and communicate an answer, working with equations has its own risks of comprehension error, just as story problems do. In effect, the authors diminish the role a problem's format can have in students' comprehension mistakes, and rather focus on how it affects the choice of a problem-solving strategy, which ends up being the crucial factor in predicting a student's accuracy.

They identify three main strategies, each with a differing success rate: \textit{symbolic manipulation}, \textit{unwind}, and \textit{guess and check}. The first, as the name suggests, is the formal method of manipulating algebraic symbols to determine the unknown value. The second strategy involves working backwards from the given equation's answer to the unknown value through a series of computations, not necessarily involving any symbolic notation. Finally, the third involves looking for the unknown by trying out different values and seeing whether the given equation makes sense, also not necessarily involving any symbolic equation. To distinguish those strategies that need algebraic notation to reach a solution from those that do not, the authors marked symbolic manipulation as a formal strategy, and unwind and guess and check as informal strategies. In the scope of their study, Nathan and Koedinger notice that the informal strategies have greater rates of success (69\% for unwind and 71\% for guess and check) than symbolic manipulation (51\%). Moreover, story problems were more frequently solved with the success-prone informal strategies than symbolic manipulation, from which they conclude that story problems are more likely to lead students to these strategies and thus to the correct answers. Therefore, symbolic problems are not inherently easier than story problems.

\subsection{Arithmetic Interference on Algebra}

While \citeA{KoedNath2004} show that the difference in success students have with story problems versus equation problems is largely due to the strategies most common for each type of problem and the success rates of those strategies, they only hint at theories of why some strategies are more successful than others and they provide no discussion of how students choose what strategy to use for a given problem. (\textbf{citation}) give such an explanation that appeals to a phenomenon suggested by (\textbf{citation of number sense}) and in effect in the problem solving realm: interference of prior knowledge. 

\subsection{Single- versus Double-Reference Algebra Problems}

Koedinger and Nathan 2008



\section{Model}

Our model attempts to synthesize the work of.... and...


\subsection{Second Level Headings}

Second level headings should be 11~point, initial caps, bold, and
flush left. Leave one line space above the heading and 1/4~line
space below the heading.

\section{Results}

Result synopsis

\subsection{Results with 2004 data}

\subsection{Results with 2008 data}

Third level headings should be 10~point, initial caps, bold, and flush
left. Leave one line space above the heading, but no space after the
heading.


\section{Discussion}

Use standard APA citation format. Citations within the text should
include the author's last name and year. If the authors' names are
included in the sentence, place only the year in parentheses, as in
\citeA{Nathan2012}, but otherwise place the entire reference in
parentheses with the authors and year separated by a comma
\citeA{Nathan2012}. List multiple references alphabetically and
separate them by semicolons
\citeA{Nathan2012,KoedNath2004}. Use the
``et~al.'' construction only after listing all the authors to a
publication in an earlier reference and for citations with four or
more authors.

Indicate footnotes with a number\footnote{Sample of the first
footnote.} in the text. Place the footnotes in 9~point type at the
bottom of the column on which they appear. Precede the footnote block
with a horizontal rule.\footnote{Sample of the second footnote.}

Number tables consecutively. Place the table number and title (in
10~point) above the table with one line space above the caption and
one line space below it, as in Table~\ref{sample-table}. You may float
tables to the top or bottom of a column, or set wide tables across
both columns.

\begin{table}[!ht]
\begin{center} 
\caption{Sample table title.} 
\label{sample-table} 
\vskip 0.12in
\begin{tabular}{ll} 
\hline
Error type    &  Example \\
\hline
Take smaller        &   63 - 44 = 21 \\
Always borrow~~~~   &   96 - 42 = 34 \\
0 - N = N           &   70 - 47 = 37 \\
0 - N = 0           &   70 - 47 = 30 \\
\hline
\end{tabular} 
\end{center} 
\end{table}


\subsection{Figures}

All artwork must be very dark for purposes of reproduction and should
not be hand drawn. Number figures sequentially, placing the figure
number and caption, in 10~point, after the figure with one line space
above the caption and one line space below it, as in
Figure~\ref{sample-figure}. If necessary, leave extra white space at
the bottom of the page to avoid splitting the figure and figure
caption. You may float figures to the top or bottom of a column, or
set wide figures across both columns.

\begin{figure}[ht]
\begin{center}
\fbox{CoGNiTiVe ScIeNcE}
\end{center}
\caption{This is a figure.} 
\label{sample-figure}
\end{figure}


\section{Acknowledgments}

Place acknowledgments (including funding information) in a section at
the end of the paper.


%\nocite{ChalnickBillman1988a}
%\nocite{Feigenbaum1963a}
%\nocite{Hill1983a}
%\nocite{OhlssonLangley1985a}
% \nocite{Lewis1978a}
%\nocite{Matlock2001}
%\nocite{NewellSimon1972a}
%\nocite{ShragerLangley1990a}


\bibliographystyle{apacite}

\setlength{\bibleftmargin}{.125in}
\setlength{\bibindent}{-\bibleftmargin}

\bibliography{CogSci_Template}


\end{document}
